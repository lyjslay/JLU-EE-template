% !TEX program = xelatex
\documentclass[openany,oneside]{book}

\usepackage{jluthesisUTF8}
%\usepackage{gbt7714}
\usepackage{amsmath}
\usepackage{fontspec}
\usepackage{graphicx}
\usepackage{tikz}
\usepackage{multirow}
\usepackage{lipsum}
\usepackage{graphicx}
\usepackage{tabularx}
\usepackage{listings}
\usepackage{algorithmic}
\usepackage{algorithm}
\usepackage{pgf}
\usepackage{booktabs}

%\usepackage{cite}
\usetikzlibrary{arrows}
\usepackage{xeCJK}







\begin{document}

\frontmatter
\sloppy % 解决中英文混排的断行问题,会加入间距,但不会影响断行 ????

% 手动在长标题中利用 \par 输入断行,
\ctitle{中文标题}
\etitle{English Titile}
% 论文 内容提要
\cthesissummary{
    中文摘要
}
% 关键词
\ckeywords{中文关键词}

\ethesissummary {
    English Abstract.
}
\ekeywords{English,Keywords}

%生成原创性声明和摘要
\makecover

\pagenumbering{Roman} 
%\pdfbookmark[0]{目~~~~录}{contents}
\tableofcontents
{\xiaosi}
%{\fontsize \fontsize{12.05pt}{14.45pt}\selectfont}
% 清除目录后面空页的页眉和页脚
\clearpage{\pagestyle{empty}\cleardoublepage}





%%% 正文
\mainmatter
\defaultfont                        % 正文使用默认字体,小四,宋体

\chapter{绪论}
\section{研究背景及意义}

研究背景及意义。研究背景及意义。研究背景及意义。研究背景及意义。研究背景及意义。研究背景及意义。研究背景及意义。研究背景及意义。\par
研究背景及意义。研究背景及意义。研究背景及意义。研究背景及意义。研究背景及意义。研究背景及意义。研究背景及意义。研究背景及意义。

\section{研究现状及挑战}

研究现状及挑战。

\section{研究内容与论文结构}

研究内容与论文结构。


\chapter{基本使用方法}

\section{公式}

插入公式,如\ref{equ:test}所示:
\begin{equation}
	\alpha = \int  \beta \cdot \gamma 
	\label{equ:test}
\end{equation}


\section{图片}

插入图片,如图\ref{fig:test}所示:
\begin{figure}[htb]
	\centering
	\includegraphics[width=0.8\linewidth]{fig/jlu}
	\caption{示例图片}
	\label{fig:test}
\end{figure}


\section{表格}

插入三线表,如表\ref{tab:test}所示,可使用Excel2LaTeX插件自动生成代码:
\begin{table}[htb]{
    \centering
    %\setlength{\belowcaptionskip}{1ex}
    \setlength{\belowcaptionskip}{1ex}
    \caption{示例表格}
    \label{tab:test}
    \setlength{\tabcolsep}{8mm}
    
    \begin{tabular}{lcccr}
        \toprule
        AA & BB & CC & DD & EE \\ 
        \midrule
        qwd & asc & asfs & asdf & fdhb \\ 
        fdgnggf  & bfgd & sggr & sdg45 & fdb45we \\
        \bottomrule
    \end{tabular}\\
}
\end{table}



\chapter{一级标题}

一级标题

\section{二级标题}

二级标题

\subsection{三级标题}

三级标题

%最后设置格式,插入参考文献。
\defaultfont
\bibliographystyle{gbt7714-2005}
\clearpage
\phantomsection
\addcontentsline{toc}{chapter}{参考文献}
\bibliography{document}
%插入致谢
\chapter*{致 \qquad 谢}
\addcontentsline{toc}{chapter}{致谢}
\thispagestyle{empty}
感谢党和国家!
\chapter*{本科期间发表论文和科研情况}
None
\addcontentsline{toc}{chapter}{本科期间发表论文和科研情况}
None
\thispagestyle{empty}

...

\end{document}
